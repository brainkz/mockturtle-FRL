% % % % % % % % % % % % % % % % % % % % % % % % % % % % % % % % % % % % % % % % % % % % % % % % 

The propagation delay incurred by the AA gates, such as splitters, may desynchronize the inputs arriving to the SA gate, producing a data hazard, as illustrated in Fig. \ref{spl_delay}.
To avoid this condition, the following constraint is added to the phase assignment ILP,
\begin{equation}
    \phi \left( g \right) \geq \phi \left( a \right)+ \left( a\notin G_{AS} \right)\forall g\in G_{SA}, a\in FI \left( g \right),
\end{equation}
where the minimum stage of gate $g$ is increased by one, if the preceding gate is not AS.

% % % % % % % % % % % % % % % % % % % % % % % % % % % % % % % % % % % % % % % % % % % % % % % % 

\subsection{Asynchronous paths}

The sharing of DFFs by a complex unclocked path has been highlighed as early as 1991 in \cite{leiserson_1991}, where sharing of DFFs is maximized to minimize the circuit area, as illustrated in Fig. \ref{fig:shared_dff}a. 
% A similar formulation is possible to estimate the number DFFs required for a path with multiple fanout
% Due to the DFF sharing in a path with multiple fanouts, the number of DFFs is determined by the fanout with the latest global phase,
% Basically, need to say that estimating the number precisely is difficult
While estimating the minimum number of DFFs in such cases as a splitter tree, is relatively simple, more complex asynchronous paths, however, require more sophisticated processing and, hence, prohibitive runtime.
% As described in section \ref{SAT_BASED_DFF_PLACEMENT}, precisely estimating the number of DFFs required by complex asynchronous path requires  processing and prohibitive runtime. 
During the phase assignment stage we therefore use an approximate number of DFFs as
\begin{equation}
    C \left( \phi \left( g \right arrow FO \left( g \right) \right) \right) = \lfloor \frac{\max\limits_{a\in FO \left( g \right)} \left( \phi \left( a \right) \right) - \phi \left( g \right)}{n} \rfloor.
    % \phi \left( a \right)+ \left( a\notin G_{AS} \right)\forall a\in FI \left( g \right),
\end{equation}
The exact number of DFFs is determined during the DFF insertion stage. 
% After considering the influence of the unclocked gates on the objective and constraint functions, t
The combined optimization problem is formulated as 

\begin{equation}\label{eq:combined_cost_func}
\min\limits_{\phi \left( g \right)\forall g\in G}
\sum\limits_{ \left( i,j \right)\in E} \lfloor \frac{\phi \left( j \right)-\phi \left( i \right) + \left( j\in G_{SA} \right)}{n} \rfloor,
\end{equation}
\begin{equation*}
\text{Subject to: }
\end{equation*}
\begin{equation}
\phi \left( i \right) <    \phi \left( j \right) \forall \left( i,j \right)\in E, j\in G_{AS},
\end{equation}
\begin{equation}
\phi \left( i \right) \leq \phi \left( j \right) \forall \left( i,j \right)\in E, j\notin G_{AS},
\end{equation}
\begin{equation}
\lfloor \frac{\phi \left( i \right)}{n} \rfloor = 0  \forall i\in I,
\end{equation}
\begin{equation}
\lfloor \frac{\phi \left( i \right)}{n} \rfloor = \lfloor \frac{\phi \left( j \right)}{n} \rfloor  \forall i,j\in O.
\end{equation}

% After completing the phase assignment, the splitters can be efficiently placed after each multifanout gate.
% while maximizing the sharing of path balancing DFFs
Each gate $g$ with $\#FO(g) > 1$ requires a splitter tree to copy the signal to each of the fanouts. 
After completing the phase assignment, a splitter tree that maximizes sharing of path balancing DFFs can be produced. 
%  can be efficiently placed after each multifanout gate.
First, the fanouts $FO(g)$ of a gate $g$ are sorted by phase in the ascending order, producing a sequence $A=[a_0, a_1,\cdots,a_k]$, $\sigma(a_i) \leq \sigma(a_j) \iff i<j$.
Next, the splitters $[s_0,...,s_{k-1}]$ are inserted at phases $a_0,\cdots,a_{k-1}$. 
The splitter $s_0$ has a gate $g$ as a fanin.
Each splitter $s_i$ has a splitter $s_{i-1}$ as a fanin, except $s_0$ whose fanin is the gate $g$. 
Each splitter $s_i$ has the $a_i$ and $s_{i+1}$ as a fanout, except $s_{k-1}$ whose fanouts are $a_{k-1}$ and $a_k$.
Using this method, the splitters are placed as late as possible within the network, maximizing sharing of the datapath.

% % % % % % % % % % % % % % % % % % % % % % % % % % % % % % % % % % % % % % % % % % % % % % % % 

(around line 427)
While the notion of stage is primarily relevant for the clocked gates, the unclocked gates are also assigned a stage.
It is however important to also assign a stage to the AA and SA elements, since a stage of an unclocked element imposes different constraints on DFF insertion process.
Consider two networks containing an asynchronous datapath illustrated in Fig. \ref{splitter_merger}.
Both networks realize the same function, however, the possible locations of path balancing DFFs are different. 
By defining the stage of the unclocked gate, the phase and epoch of the path balancing DFFs within the datapath can be determined. 






